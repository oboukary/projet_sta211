\documentclass[11pt,a4paper, x11names]{article}\usepackage[]{graphicx}\usepackage[]{color}
% maxwidth is the original width if it is less than linewidth
% otherwise use linewidth (to make sure the graphics do not exceed the margin)
\makeatletter
\def\maxwidth{ %
  \ifdim\Gin@nat@width>\linewidth
    \linewidth
  \else
    \Gin@nat@width
  \fi
}
\makeatother

\definecolor{fgcolor}{rgb}{0, 0, 0}
\makeatletter
\@ifundefined{AddToHook}{}{\AddToHook{package/xcolor/after}{\definecolor{fgcolor}{rgb}{0, 0, 0}}}
\makeatother
\newcommand{\hlnum}[1]{\textcolor[rgb]{0,0,1}{\textbf{#1}}}%
\newcommand{\hlstr}[1]{\textcolor[rgb]{0.639,0.082,0.082}{#1}}%
\newcommand{\hlcom}[1]{\textcolor[rgb]{0.376,0.376,0.376}{#1}}%
\newcommand{\hlopt}[1]{\textcolor[rgb]{0,0,0}{#1}}%
\newcommand{\hlstd}[1]{\textcolor[rgb]{0,0,0}{#1}}%
\newcommand{\hlkwa}[1]{\textcolor[rgb]{0,0,1}{#1}}%
\newcommand{\hlkwb}[1]{\textcolor[rgb]{0,0,1}{#1}}%
\newcommand{\hlkwc}[1]{\textcolor[rgb]{0.169,0.569,0.686}{#1}}%
\newcommand{\hlkwd}[1]{\textcolor[rgb]{0.169,0.569,0.686}{#1}}%
\let\hlipl\hlkwb

\usepackage{framed}
\makeatletter
\newenvironment{kframe}{%
 \def\at@end@of@kframe{}%
 \ifinner\ifhmode%
  \def\at@end@of@kframe{\end{minipage}}%
  \begin{minipage}{\columnwidth}%
 \fi\fi%
 \def\FrameCommand##1{\hskip\@totalleftmargin \hskip-\fboxsep
 \colorbox{shadecolor}{##1}\hskip-\fboxsep
     % There is no \\@totalrightmargin, so:
     \hskip-\linewidth \hskip-\@totalleftmargin \hskip\columnwidth}%
 \MakeFramed {\advance\hsize-\width
   \@totalleftmargin\z@ \linewidth\hsize
   \@setminipage}}%
 {\par\unskip\endMakeFramed%
 \at@end@of@kframe}
\makeatother

\definecolor{shadecolor}{rgb}{.97, .97, .97}
\definecolor{messagecolor}{rgb}{0, 0, 0}
\definecolor{warningcolor}{rgb}{1, 0, 1}
\definecolor{errorcolor}{rgb}{1, 0, 0}
\makeatletter
\@ifundefined{AddToHook}{}{\AddToHook{package/xcolor/after}{
\definecolor{shadecolor}{rgb}{.97, .97, .97}
\definecolor{messagecolor}{rgb}{0, 0, 0}
\definecolor{warningcolor}{rgb}{1, 0, 1}
\definecolor{errorcolor}{rgb}{1, 0, 0}
}}
\makeatother
\newenvironment{knitrout}{}{} % an empty environment to be redefined in TeX

\usepackage{alltt}
\usepackage[utf8]{inputenc}
\usepackage[french]{babel}
\usepackage[left=2cm,right=2cm,top=2cm,bottom=2cm]{geometry}
\usepackage[colorlinks=true,linkcolor=blue,anchorcolor=black,citecolor=black,filecolor=black,menucolor=black,runcolor=black,urlcolor=black]{hyperref} 

\usepackage[T1]{fontenc} % Font encoding
\usepackage{graphicx}
\graphicspath{{Images/}}
\usepackage{eso-pic} 
\usepackage{subfig} 
\usepackage{caption} 
\usepackage{tabu}
\usepackage{longtable}
\usepackage{transparent}
\usepackage{amsmath}
\usepackage{amsthm}
\usepackage{bm}
\usepackage{mdframed}
\usepackage[overload]{empheq}  
\usepackage{tabularx}
\usepackage{longtable}
\usepackage{colortbl}
\usepackage{cleveref}
\usepackage[square, numbers, sort&compress]{natbib} 
\bibliographystyle{plain} 
\usepackage{appendix}
\usepackage{enumitem}
\usepackage{amsthm,thmtools,xcolor} 
\usepackage{comment} % Comment part of code
\usepackage{fancyhdr} % Fancy headers and footers
\usepackage{tcolorbox} 
\usepackage{amsmath,amsfonts,amssymb}
\usepackage{pgfplots,tikz}
\usetikzlibrary{matrix}
\usepackage{enumitem}  
\usepackage[normalem]{ulem}
%fonts change 
\usepackage[charter]{mathdesign}
\usepackage{fancybox,framed}
\tcbuselibrary{skins,breakable,xparse}
\usepackage{shadowtext}
\usepackage{titlesec}
\usepackage{setspace}

\usepackage{booktabs}
\usepackage{caption}
\usepackage{float}
\usepackage{titlesec}
\usepackage{capt-of}

%dashed line
\usepackage{array}
\usepackage{arydshln}
\setlength\dashlinedash{0.2pt}
\setlength\dashlinegap{1.5pt}
\setlength\arrayrulewidth{0.3pt}

%Widows & Orphans & Penalties

\widowpenalty500
\clubpenalty500
\clubpenalty=9996
\exhyphenpenalty=50 %for line-breaking at an explicit hyphen
\brokenpenalty=4991
\predisplaypenalty=10000
\postdisplaypenalty=1549
\displaywidowpenalty=1602
\floatingpenalty = 20000
\setstretch{1.5}
\pagestyle{fancy}
\renewcommand{\footrulewidth}{0pt}
\renewcommand{\headrulewidth}{1pt}
\setlist[enumerate,1]{label=\arabic*)}
\setlength{\parindent}{0mm}
\titleformat{\section}
{\normalfont\Large\bfseries}{\thesection .}{1.5mm}{}
\titleformat{\subsection}
{\normalfont\large\bfseries}{\thesubsection}{1.5mm}{}
\titleformat{\subsubsection}
{\normalfont\normalsize\bfseries}{\thesubsubsection}{1.5mm}{}
\titleformat{\paragraph}[runin]
{\normalfont\normalsize\bfseries}{\theparagraph}{1.5mm}{}
\titleformat{\subparagraph}[runin]
{\normalfont\normalsize\bfseries}{\thesubparagraph}{1.5mm}{}

\renewcommand{\d}{\,\mathrm{d}}    

\renewcommand{\title}{Certificat de spécialisation analyste de données massives}
% -> author name and surname
\newcommand{\authora}{Boukary OUEDRAOGO}
\newcommand{\authorb}{Bangaly CAMARA}
\newcommand{\authorc}{Imad EL HAMMA}
% -> MSc course
\newcommand{\course}{STA211}
\newcommand{\advisor}{Mme Niang}
% IF AND ONLY IF you need to modify the co-supervisors you also have to modify the file Configuration_files/title_page.tex (ONLY where it is marked)
\newcommand{\firstcoadvisor}{Name Surname} % insert if any otherwise comment
\newcommand{\secondcoadvisor}{Name Surname} % insert if any otherwise comment
% -> author ID
\newcommand{\ID}{Student ID}
% -> academic year
\newcommand{\YEAR}{2021-2022}
\definecolor{bluePoli}{rgb}{0.87, 0.36, 0.51}
\definecolor{babyblueeyes}{rgb}{0.63, 0.79, 0.95}
\definecolor{blush}{rgb}{0.87, 0.36, 0.51}
\definecolor{bubblegum}{rgb}{0.99, 0.76, 0.8}
\definecolor{charcoal}{rgb}{0.21, 0.27, 0.31}
% Custom theorem environments
\declaretheoremstyle[
  headfont=\color{bluePoli}\normalfont\bfseries,
  bodyfont=\color{black}\normalfont\itshape,
]{colored}

\captionsetup[figure]{labelfont={color=charcoal}} % Set colour of the captions
\captionsetup[table]{labelfont={color=charcoal}} % Set colour of the captions
\captionsetup[algorithm]{labelfont={color=charcoal}} % Set colour of the captions

\theoremstyle{colored}
\newtheorem{theorem}{Theorem}[section]
\newtheorem{proposition}{Proposition}[section]

% Enhances the features of the standard "table" and "tabular" environments.
\newcommand\T{\rule{0pt}{2.6ex}}
\newcommand\B{\rule[-1.2ex]{0pt}{0pt}}

% Algorithm description
\newcounter{algsubstate}
\renewcommand{\thealgsubstate}{\alph{algsubstate}}
\newenvironment{algsubstates}{
  \setcounter{algsubstate}{0}%
  \renewcommand{\STATE}{%
    \stepcounter{algsubstate}%
    \Statex {\small\thealgsubstate:}\space}
}{}

% Custom theorem environment
\newcolumntype{L}[1]{>{\raggedright\let\newline\\\arraybackslash\hspace{0pt}}m{#1}}
  \newcolumntype{C}[1]{>{\centering\let\newline\\\arraybackslash\hspace{0pt}}m{#1}}
    \newcolumntype{R}[1]{>{\raggedleft\let\newline\\\arraybackslash\hspace{0pt}}m{#1}}
      
      % Custom itemize environment
      \setlist[itemize,1]{label=$\bullet$}
      \setlist[itemize,2]{label=$\circ$}
      \setlist[itemize,3]{label=$-$}
      \setlist{nosep}
      
      % Create command for background pic
      \newcommand\BackgroundPic{% Adding background picture
        \put(237,365){
          \parbox[b][\paperheight]{\paperwidth}{%
            \vfill
            \centering
            \transparent{0.4}
            \includegraphics[width=0.44\paperwidth]{raggiera_polimi.eps}%
            \vfill}
        }
      }
      
      % Set indentation
      \setlength\parindent{0pt}
      
      % Custom title commands
      \titleformat{\section}
      {\color{charcoal}\normalfont\Large\bfseries}
      {\color{charcoal}\thesection.}{1em}{}
      \titlespacing*{\section}
      {0pt}{3.3ex}{3.3ex}
      
      \titleformat{\subsection}
      {\color{charcoal}\normalfont\large\bfseries}
      {\color{charcoal}\thesubsection.}{1em}{}
      \titlespacing*{\subsection}
      {0pt}{3.3ex}{3.3ex}
      
      % Custom headers and footers
      \pagestyle{fancy}
      \fancyhf{}
      
      \fancyfoot{}
      \fancyfoot[C]{\thepage} % page
      \renewcommand{\headrulewidth}{0mm} % headrule width
      \renewcommand{\footrulewidth}{0mm} % footrule width
      
      \makeatletter
      \patchcmd{\headrule}{\hrule}{\color{black}\hrule}{}{} % headrule
      \patchcmd{\footrule}{\hrule}{\color{black}\hrule}{}{} % footrule
      \makeatother
\IfFileExists{upquote.sty}{\usepackage{upquote}}{}
\begin{document}
\begin{titlepage}
\thispagestyle{empty}

\begin{tikzpicture}[remember picture,overlay]
\node at (current page.south west)
{\begin{tikzpicture}[remember picture, overlay]
  \shade[bottom color=bubblegum,top color=white] (0,0) rectangle
  (\paperwidth,.7\paperheight);
  \node [color=gray!50,rotate=-20]at (0.35\textwidth,9){\resizebox{7cm}{1.5cm}{$\displaystyle u(x)\approx U_h=\sum_{j=1}^{N}c_j\varphi_j(x)$}};
  \node [color=gray!50,rotate=-5]at (.8\textwidth,5){\resizebox{9cm}{1.5cm}{$ \displaystyle u''(x_i ) \approx\dfrac{1}{h^2}\left[u(x_{ i+1} ) - 2u(x_i ) + u(x_{i-1 } )\right] $}
  };
  \end{tikzpicture}
};

\end{tikzpicture}
\vspace{-1cm}
%-----------------------------------------------------------------------------------
  %  page de garde
%-----------------------------------------------------------------------------------
  \begin{center}

\begin{large}
\includegraphics[scale=0.5]{Images/logocnam.png} \\
\textsc{Certificat de spécialisation analyste de données massives}\\ \bigskip
\textbf{ Entreposage et fouille de données
} \end{large}
\end{center}
\bigskip
\shadowoffset{3pt}
\begin{tcolorbox}[blanker,top=1.2cm,bottom=1.2cm,borderline horizontal={4pt}{0pt}{charcoal},colupper=bluePoli]
\begin{center}
\shadowtext{\resizebox{\textwidth}{1.5cm}{\textbf{Projet de STA211} }}
\end{center}
\end{tcolorbox}


\bigskip

\begin{minipage}{0.4\textwidth}
\large
\emph{Auteurs:}\\
\authora\\[3mm]
\authorb\\[3mm]
\authorc\\[3mm]
\end{minipage}
\hfill
\begin{minipage}{0.4\textwidth}
 \large
\emph{Professeur:}\\
\advisor\\[3mm]
\\[3mm]
\\[3mm]
\end{minipage}\\[2cm] 


\textcolor{charcoal}{\rule{.82\textwidth}{4pt}\hfill {\fontfamily{pzc}\fontsize{.7cm}{0cm}\selectfont \YEAR }}
\end{titlepage}
\begin{abstract}
\end{abstract}
\section{Introduction}

%--------------------------------------------------------------------------------------------------
%             00-  MISE EN PLACE DE L'ENVIRONNEMENT SWEAVE POUR LA GENERATION DU DOCUMENT PDF
%--------------------------------------------------------------------------------------------------


%--------------------------------------------------------------------------------------------------
%                   01- CHARGEMENT DES PACKAGES R UTILES POUR LE TRAVAIL
%--------------------------------------------------------------------------------------------------

%--------------------------------------------------------------------------------------------------
%                     01- SECTION 
%--------------------------------------------------------------------------------------------------
\section{Chargement des données et analyses préliminaires}
Cette partie est consacrée au chargement des données et à la sélection des variables 
liées au ménage. Certaines variables qui sont codées comme des variables numériques mais qui en réalité sont qualitatives seront recordées en variables facteurs.  
\subsection{Chargement des données}

%--------------------------------------------------------------------------------------------------
% IMPORTATION DE LA BASE DE DONNEES ET CREATION DE SOUS BASES DE TRAVAIL
%--------------------------------------------------------------------------------------------------


%--------------------------------------------------------------------------------------------------
%                 03- ANALYSE UNIVARIEE
%--------------------------------------------------------------------------------------------------
Après le chargement des données, l'étape suivante est l'analyse univariée. 
On peut regarder les statistiques descriptives simples avec la function \textbf{summary} et la fonction \textbf{describe}.

\begin{knitrout}
\definecolor{shadecolor}{rgb}{1, 1, 1}\color{fgcolor}\begin{kframe}
\begin{verbatim}
## Description of .
\end{verbatim}
\end{kframe}
\end{knitrout}

%--------------------------------------------------------------------------------------------------
%                         04- ANALYSE BIVARIEE
%----------------------------------------------------------------------------
%-------------------------------------------------------------------------------
%  SECTION 1: Variables quantitatives
%-------------------------------------------------------------------------------
\subsection{Analyse descriptive des variables quantitatives }
\subsubsection{Analyse univarié}

\begin{table*}[!h] \centering
%\ra{1.3}
\begin{small}
\begin{tabular}{@{}lrrrrrr@{}}\toprule
\textbf{Variables}& \textbf{Moyenne} & \textbf{Médiane}& \textbf{Min} & \textbf{Maximum} & \textbf{Variance} & \textbf{Ecart-type} \\ \midrule
\textbf{Age}          & 51.87 &  52 & 18 & 89 & 209.91 & 14.49\ \\ \hdashline
\textbf{Revenus}      & 2~573.20 & 2~349 & 535&7~600 & 1~967~644.23&1~402.73   \\ \hdashline
\textbf{Nombre d'enfants dont âge > 10 ans} &  0.44 & 0 & 0 & 3 & 0.67 & 0.82  \\  \hdashline
\textbf{Nombre d'enfants dont âge <= 10 ans} &  0.32 & 0 & 0 & 3 & 0.48 & 0.70 \\
\bottomrule
\end{tabular}
\end{small}
\caption{Statistiques descriptives des variables quantitatives}
\end{table*}



%--------------------------------------------------------------------------------------------------
%                         05- ANALYSE FACTORIELLE
%--------------------------------------------------------------------------------------------------


\par Etant donné la structure en groupe des données, nous allons réaliser des analyses factorielles sur
chaque groupe de variables. Pour le bloc chaque bloc(ménage, habitude, logement), nous réalisons des \textbf{ACP} sur les variables quantitatives, une \textbf{ACM} sur les variables qualitatives et une \textbf{AFM} sur chaque bloc. L'objectif est de voir s'il existe une différence entre individus par rapport à chaque groupe de variable ou s'il existe des groupes d'individus homogènes par rapport à chaque groupe ou chaque bloc de variables.
%%%%%%%%%%%%%%%%%%%%%%%%%%%%%%%%%%%%%%%%%%%%%%%%%%%%%%%%%%%%%%%%%%%%%%%%%
%  MENAGE
%%%%%%%%%%%%%%%%%%%%%%%%%%%%%%%%%%%%%%%%%%%%%%%%%%%%%%%%%%%%%%%%%%%%%%%%

\subsubsection{ACP sur les variables quantitatives du bloc ménage}
\begin{minipage}{0.49\linewidth}
\begin{mdframed}
\begin{knitrout}
\definecolor{shadecolor}{rgb}{1, 1, 1}\color{fgcolor}
\includegraphics[width=\maxwidth]{figure/unnamed-chunk-2-1} 
\end{knitrout}
\end{mdframed}
\end{minipage}
\hfill
\begin{minipage}{0.49\linewidth}
\begin{mdframed}
\begin{knitrout}
\definecolor{shadecolor}{rgb}{1, 1, 1}\color{fgcolor}
\includegraphics[width=\maxwidth]{figure/unnamed-chunk-3-1} 
\end{knitrout}
\end{mdframed}
\end{minipage}

D'après les résultats de l'ACP sur les variables quantitatives du bloc ménage,
les deux premières dimensions portent 61.7 de la variabilité contenu dans ce groupe de variables. Cela répresente un part significative. Les variables \textbf{NBPerson},\textbf{NBEnftssup0} et \textbf{NBEnftsinf10} sont fortement corrélés au premier axe factoriel de l'ACP. A l'opposé, les variables  \textbf{Age} et \textbf{Revenus} sont fortement corrélées au deuxième axe factoriel de l'ACP. Les individus(ménages) qui ont de fortes coordonnées sur l'axe 2 sont des ménages relativement aisés et âgés. Ceux qui ont de fortes coordonnées sur l'axe 1 sont des ménages composés de beaucoup de personnes et avec des enfants.

\subsubsection{ACM sur les variables qualitatives du bloc ménage}
\begin{minipage}{0.49\linewidth}
\begin{mdframed}
\begin{knitrout}
\definecolor{shadecolor}{rgb}{1, 1, 1}\color{fgcolor}
\includegraphics[width=\maxwidth]{figure/unnamed-chunk-4-1} 
\end{knitrout}
\end{mdframed}
\end{minipage}
\hfill
\begin{minipage}{0.49\linewidth}
\begin{mdframed}
\begin{knitrout}
\definecolor{shadecolor}{rgb}{1, 1, 1}\color{fgcolor}
\includegraphics[width=\maxwidth]{figure/unnamed-chunk-5-1} 
\end{knitrout}
\end{mdframed}
\end{minipage}
\vfill
\begin{minipage}{0.49\linewidth}
\begin{mdframed}
\begin{knitrout}
\definecolor{shadecolor}{rgb}{1, 1, 1}\color{fgcolor}
\includegraphics[width=\maxwidth]{figure/unnamed-chunk-6-1} 

\includegraphics[width=\maxwidth]{figure/unnamed-chunk-6-2} 
\end{knitrout}
\end{mdframed}
\end{minipage}
\hfill
\begin{minipage}{0.49\linewidth}
\begin{mdframed}
\begin{knitrout}
\definecolor{shadecolor}{rgb}{1, 1, 1}\color{fgcolor}
\includegraphics[width=\maxwidth]{figure/unnamed-chunk-7-1} 
\end{knitrout}
\end{mdframed}
\end{minipage}

\subsubsection{AFM sur les variable du bloc ménage}
\begin{minipage}{0.49\linewidth}
\begin{mdframed}
\begin{knitrout}
\definecolor{shadecolor}{rgb}{1, 1, 1}\color{fgcolor}
\includegraphics[width=\maxwidth]{figure/unnamed-chunk-8-1} 
\end{knitrout}

\end{mdframed}
\end{minipage}


%%%%%%%%%%%%%%%%%%%%%%%%%%%%%%%%%%%%%%%%%%%%%%%%%%%%%%%%%%%%%%%%%%%%%%%%%
%  LOGEMENT
%%%%%%%%%%%%%%%%%%%%%%%%%%%%%%%%%%%%%%%%%%%%%%%%%%%%%%%%%%%%%%%%%%%%%%%%

\subsubsection{ACP sur les variables quantitatives du bloc ménage}
\begin{minipage}{0.49\linewidth}
\begin{mdframed}
\begin{knitrout}
\definecolor{shadecolor}{rgb}{1, 1, 1}\color{fgcolor}
\includegraphics[width=\maxwidth]{figure/unnamed-chunk-9-1} 
\end{knitrout}
\end{mdframed}
\end{minipage}
\hfill
\begin{minipage}{0.49\linewidth}
\begin{mdframed}
\begin{knitrout}
\definecolor{shadecolor}{rgb}{1, 1, 1}\color{fgcolor}
\includegraphics[width=\maxwidth]{figure/unnamed-chunk-10-1} 
\end{knitrout}
\end{mdframed}
\end{minipage}

D'après les résultats de l'ACP sur les variables quantitatives du bloc ménage,
les deux premières dimensions portent 61.7 de la variabilité contenu dans ce groupe de variables. Cela répresente un part significative. Les variables \textbf{NBPerson},\textbf{NBEnftssup0} et \textbf{NBEnftsinf10} sont fortement corrélés au premier axe factoriel de l'ACP. A l'opposé, les variables  \textbf{Age} et \textbf{Revenus} sont fortement corrélées au deuxième axe factoriel de l'ACP. Les individus(ménages) qui ont de fortes coordonnées sur l'axe 2 sont des ménages relativement aisés et âgés. Ceux qui ont de fortes coordonnées sur l'axe 1 sont des ménages composés de beaucoup de personnes et avec des enfants.

\subsubsection{ACM sur les variables qualitatives du bloc ménage}
\begin{minipage}{0.49\linewidth}
\begin{mdframed}
\begin{knitrout}
\definecolor{shadecolor}{rgb}{1, 1, 1}\color{fgcolor}
\includegraphics[width=\maxwidth]{figure/unnamed-chunk-11-1} 
\end{knitrout}
\end{mdframed}
\end{minipage}
\hfill
\begin{minipage}{0.49\linewidth}
\begin{mdframed}
\begin{knitrout}
\definecolor{shadecolor}{rgb}{1, 1, 1}\color{fgcolor}
\includegraphics[width=\maxwidth]{figure/unnamed-chunk-12-1} 
\end{knitrout}
\end{mdframed}
\end{minipage}
\vfill
\begin{minipage}{0.49\linewidth}
\begin{mdframed}
\begin{knitrout}
\definecolor{shadecolor}{rgb}{1, 1, 1}\color{fgcolor}
\includegraphics[width=\maxwidth]{figure/unnamed-chunk-13-1} 

\includegraphics[width=\maxwidth]{figure/unnamed-chunk-13-2} 
\end{knitrout}
\end{mdframed}
\end{minipage}
\hfill
\begin{minipage}{0.49\linewidth}
\begin{mdframed}
\begin{knitrout}
\definecolor{shadecolor}{rgb}{1, 1, 1}\color{fgcolor}
\includegraphics[width=\maxwidth]{figure/unnamed-chunk-14-1} 
\end{knitrout}
\end{mdframed}
\end{minipage}

\subsubsection{AFM sur les variable du bloc ménage}
\begin{minipage}{0.49\linewidth}
\begin{mdframed}
\begin{knitrout}
\definecolor{shadecolor}{rgb}{1, 1, 1}\color{fgcolor}
\includegraphics[width=\maxwidth]{figure/unnamed-chunk-15-1} 
\end{knitrout}

\end{mdframed}
\end{minipage}


%-------------------------------------------------------------------------------
%   SECTION 2: Variables qualitatives
%-------------------------------------------------------------------------------

\end{document}

